% !TEX root =  main.tex

\subsection{La rete Peer-To-Peer -- Jolella}
\label{subsec:jolella}

Jolella, ispirata al programma GNUtella~\footnote{Creato nel
 2000, GNUtella è stato uno dei primi protocolli di condivisione di files su una
 rete decentralizzata peer-to-peer.}~\cite{gnutella}, è una rete decentralizzata per la
condivisione di files tra nodi della rete stessa (anche detta \textit{P2P}). Il
termine peer-to-peer si riferisce ad una classe di sistemi ed applicazioni che
utilizzano risorse distribuite per eseguire una funzionalità (critica) in modo
decentralizzato~\cite{peer-to-peer}.

Ogni nodo in una rete P2P viene chiamato generalmente \textbf{peer}, ha la
medesima importanza, e svolge le stesse funzioni di tutti gli altri nodi.

Le caratteristiche di un sistema P2P -- decentralizzata -- possono essere
riassunte nelle seguenti:

\begin{itemize}

 \item i peer sono indipendenti (autonomi);

 \item ogni peer ha funzioni di \textbf{client} e di \textbf{server}
       contemporaneamente, e condivide delle risorse (in questo caso files);

 \item possono essere presenti dei nodi con funzionalità diverse rispetto agli
       altri, ad esempio per il monitoraggio o, in generale, per offrire operazioni
       specifiche;

 \item il sistema è \textbf{altamente distribuito}, il numero di peers può
       essere dell'ordine delle centinaia di migliaia;

 \item il sistema è \textbf{altamente dinamico}, un peer può entrare ed uscire
       dalla rete in ogni momento.

       \textbf{N.B.}

       \begin{itemize}

        \item le operazioni di ingresso/uscita (\textbf{join}/\textbf{leave}) dalla
              rete fanno parte dell'interfaccia di ogni peer.

       \end{itemize}

\end{itemize}

\paragraph{Una rete P2P per la condivisione di files.} La condivisione di files
costituisce il principale utilizzo delle reti decentralizzate fino dagli anni
`90 (si parla di quasi il $30\%$ della rete Internet). Perchè una rete P2P sia
abilitata alla condivisione di files è necessario che ogni nodo esponga delle
funzionalità di \textbf{pubblicazione}, \textbf{ricerca} e \textbf{recupero}.

\paragraph{Sistemi con ricerca flood-based}.


\subsection{Il nodo/peer -- il Jeer}
\label{subsec:jeer}

Un peer della rete Jolella (che, per disambiguità, chiameremo \textbf{Jeer}) si
occupa sia delle operazioni di gestione della rete, sia delle funzionalità del
peer (descritte entrambe nella sezione~\ref{subsec:jolella}).

\subsection{Il file di Jolella -- il Jile}



\subsection{Il monitoraggio della rete -- il Network Monitor}
