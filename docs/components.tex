% !TEX root =  main.tex

In questa sezione vengono descritte le componenti del sistema preso in
considerazione per la realizzazione del progetto Jolella. Tali descrizioni hanno
il solo scopo di informare, e non di porre vincoli sulle scelte implementative.
Alcuni di questi vincoli e limitazioni sono invece discussi nella
sezione~\ref{sec:implementation}, insieme alle motivazioni delle stesse.

\subsection{La rete Peer-To-Peer}
\label{subsec:p2p}

Il termine peer-to-peer~\cite{AS04,Balakrishnan:2003:LUD:606272.606299} si
riferisce ad una classe di sistemi ed applicazioni che utilizzano risorse
distribuite per eseguire una funzionalità (critica) in modo
decentralizzato~\cite{peer-to-peer}.

Ogni nodo in una rete P2P viene chiamato generalmente \textbf{peer}, ha la
medesima importanza e svolge le stesse funzioni di tutti gli altri nodi.

Le caratteristiche di un sistema P2P possono essere riassunte nelle seguenti:

\begin{itemize}

 \item i peer sono indipendenti (autonomi);

 \item ogni peer ha funzioni di \textbf{client} e di \textbf{server}
       contemporaneamente, e condivide delle risorse (in questo caso files);

 \item possono essere presenti dei nodi con funzionalità diverse rispetto agli
       altri, ad esempio per il monitoraggio o, in generale, per offrire operazioni
       specifiche;

 \item il sistema è \textbf{altamente distribuito}, il numero di peers può
       essere dell'ordine delle centinaia di migliaia;

 \item il sistema è \textbf{altamente dinamico}, un peer può entrare ed uscire
       dalla rete in ogni momento.

       \textbf{N.B.}

       \begin{itemize}

        \item le operazioni di ingresso/uscita (\textbf{join}/\textbf{leave}) dalla
              rete fanno parte dell'interfaccia di ogni peer.

       \end{itemize}

\end{itemize}

\subsubsection{Architetture di una rete P2P}

Esistono tre tipologie di reti distribuite peer-to-peer:

\begin{enumerate}

 \item Decentralizzate pure -- dove tutti i nodi sono peer, non esiste nessun
       coordinatore centralizzato. Ogni peer può funzionare come client e/o server.

 \item Parzialmente centralizzate -- in queste reti alcuni nodi (supernodi)
       facilitano l’interconnessione tra i peer, questi hanno, di solito, un indice
       locale centralizzato dei peer locali. Il
       pattern di comunicazione è: un peer comunica con il supernodo, che a sua volta
       comunica con gli altri peer.

 \item Decentralizzate ibride -- queste reti presentano un server centralizzato
       che facilita l’interazione tra i peer (servizio di \textbf{localizzazione}, ad
       esempio tramite \textbf{location}).

\end{enumerate}

\paragraph{Una rete P2P per la condivisione di files.} La condivisione di files
costituisce il principale utilizzo delle reti decentralizzate fin dagli anni
`90 (si parla di quasi il $30\%$ della rete Internet). Perchè una rete P2P sia
abilitata alla condivisione di files è necessario che ogni nodo esponga delle
funzionalità di \textbf{pubblicazione}, \textbf{ricerca} e \textbf{recupero}.

Vediamo un ``classico'' esempio di comunicazione tra due entità in una rete
peer-to-peer per la condivisione di un file:

\begin{enumerate}

 \item Alice esegue un \textbf{Client} P2P;

 \item Alice registra il suo contenuto nel sistema P2P;

 \item Alice cerca ``Hey Jude'';

 \item Il \textbf{Client} visualizza altri peer che hanno una copia di ``Hey
       Jude'';

 \item Alice sceglie uno, o più peer, per esempio Bob;

 \item il file viene copiato da Bob ad Alice;

 \item mentre Alice esegue il download, altri peer chiedono ad Alice l'upload
       di file.

\end{enumerate}

L'esempio riportato, senza pretesa di descrivere in modo completo il flusso di
comunicazioni, evidenzia alcune importanti proprietà che un \textbf{Client} di
una rete P2P per la condivisione files deve soddisfare. Esso, infatti, non solo
consente ad Alice di registrare un file per la sua condivisione ma, ``sotto il
tappeto'', si occupa, per esempio, di fare il \textit{join} della rete prima di
tutto il resto. Inoltre, permette ad Alice di copiare files condivisi da altri
utenti della rete e consente di individuare quali peer possiedono un determinato
file, tramite una ricerca basata su \textbf{keyword}.

\paragraph{Sistemi con ricerca flood-based.} Riguardo all'ultimo punto trattato,
cioè quello della ricerca in una rete decentralizzata, tale problema può essere
ricondotto a quello del \textit{routing} o instradamento.


\subsection{Il nodo/peer}


\subsection{Il monitor di rete}
