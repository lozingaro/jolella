% !TEX root =  main.tex

Globalmente, vengono consegnati due prodotti: la documentazione e l'implementazione del progetto. La valutazione
finale avviene mediante una discussione di gruppo.

\begin{tcolorbox}[colback=yellow!20!white,colframe=yellow!75!black,title=\textbf{N.B.}]
 \begin{enumerate}
  \item non si accettano richieste di eccezioni sui progetti con motivazioni legate a esigenze di laurearsi o di non voler pagare le tasse per un altro anno.
  \item chi copia o fa copiare, anche solo in parte, si vede invalidare completamente il progetto senza possibilità di appello $\rightarrow$ deve quindi rifare un nuovo progetto l'anno successivo.
 \end{enumerate}
\end{tcolorbox}

\subsection{La documentazione}
È possibile scrivere la documentazione nel formato preferito, l'importante è che il PDF generato rispetti la struttura del modello (riportato qui in basso). La documentazione ha lunghezza di quattro o cinque pagine (quindi da 8 a 10 facciate), è scritto con font di grandezza \textbf{12pt} e viene consegnato in formato PDF.
Di seguito viene riportato un esempio di documentazione con le principali caratteristiche da inserire.
\begin{tcolorbox}[colback=green!20!white,colframe=green!75!black,title=L'intestazione della Documentazione]
 \begin{itemize}
  \item Jollar -- Laboratorio Sistemi Operativi A.A. 2017-2018
  \item Nome del Gruppo
  \item Indirizzo mail di riferimento: nome.cognome@studio.unibo.it
  \item Componenti:
        \begin{itemize}
         \item Cognome, Nome, Matr. 0000424242
         \item \dots
        \end{itemize}
 \end{itemize}
\end{tcolorbox}
\begin{tcolorbox}[colback=green!20!white,colframe=green!75!black,title=Il corpo della Documentazione]
 \begin{enumerate}
  \item Descrizione generale del progetto -- descrizione delle features implementate e del contenuto della documentazione.
  \item Istruzioni per la demo -- le istruzioni per eseguire una demo.
  \item Discussione sulle strategie di implementazione
        \begin{enumerate}
         \item Struttura del progetto -- come è stato diviso il progetto, perché, i problemi principali riscontrati, le alternative considerate e le soluzioni scelte.
         \item Sezione di descrizione della feature x -- abbiamo implementato la funzione di `foo` \dots (con esempi di codice).
        \end{enumerate}
 \end{enumerate}
\end{tcolorbox}

\subsubsection{Griglia di Valutazione}
La valutazione della documentazione verte sull'analisi dello scritto e sulla sua capacità di esprimere con chiarezza i concetti descritti, soprattutto grazie all'uso di esempi. In particolare la griglia di valutazione usata è la seguente:
\newpage
\begin{table}[ht]
 \centering
 {\small
  \begin{tabular}{|l|l|}
   \hline
   \textit{\begin{tabular}[c]{@{}l@{}}Qualità dell'informazione\end{tabular}} & \begin{tabular}[c]{@{}l@{}}Riconoscimento dei problemi (di concorrenza)\\e loro descrizione.\end{tabular} \\ \hline
   \textit{\begin{tabular}[c]{@{}l@{}}Uso degli esempi\end{tabular}} & \begin{tabular}[c]{@{}l@{}}Presenza di almeno un esempio in tutte le scelte\\implementative.\end{tabular} \\ \hline
   \textit{\begin{tabular}[c]{@{}l@{}}Analisi delle scelte \\ implementative\end{tabular}} & \begin{tabular}[c]{@{}l@{}}Descrizione  della propria scelta implementativa e\\presenza di proposte di alternative valide.\end{tabular} \\ \hline
  \end{tabular}
 }
\end{table}

\subsection{L'Implementazione}

Il progetto viene sviluppato utilizzando il linguaggio Jolie. Non ci sono
requisiti riguardo ai protocolli (\textit{protocol}) e i media
(\textit{location}) utilizzati per realizzare la comunicazione tra i componenti
del sistema. La gestione del progetto avviene col supporto del sistema
\textit{git}.  Il codice del progetto è contenuto in un repository su
\href{http://gitlab.com}{GitLab} e viene gestito seguendo la procedura descritta
sotto.

%
\begin{tcolorbox}[colback=blue!20!white,colframe=blue!75!black,title=GitLab]
 \begin{itemize}
  \item Ogni membro del gruppo crea un account su GitLab;
  \item il referente del gruppo  crea un nuovo progetto cliccando sul \textbf{+} in alto a destra nella schermata principale di GitLab, inserendo il nome ``LabSO\_NomeGruppo'' e cliccando su \textbf{New Project};
  \item una volta che il progetto è stato creato, il referente aggiunge ogni membro del gruppo come \textbf{role permission} $>$\textbf{Developer} al progetto andando su \textbf{Settings} $>$ \textbf{Members} nel menù a sinistra, cercandoli in base allo username registrato su GitLab;
  \item il referente aggiunge l'utente ``stefanopiozingaro'' come \textbf{role permission} $>$ \textbf{Reporter}.
 \end{itemize}
\end{tcolorbox}
%
Al momento della consegna, il repository dovrà contenere i sorgenti del progetto e la relazione, nominata \textbf{REPORT\_LSO.pdf}. Per effettuare la consegna:
\begin{enumerate}
 \item nella pagina di GitLab del repository, cliccare sulle voci del menù \textbf{Repository} $>$ \textbf{Tags} $>$ \textbf{New Tag};
 \item digitare come \textbf{Tag Name} il nome \textbf{Consegna};
 \item cliccare su \textbf{Create Tag} per eseguire la creazione del \textbf{Tag} di consegna.
\end{enumerate}
%
Una volta creato il Tag, inviare una email di notifica di consegna con soggetto \textbf{CONSEGNA LSO - NOME GRUPPO} a \href{stefanopio.zingaro@unibo.it}{questo indirizzo di posta elettronica}.
\begin{tcolorbox}[colback=yellow!20!white,colframe=yellow!75!black,title=\textbf{N.B.}]
 La documentazione va consegnata in forma cartacea nella casella del prof. Sangiorgi al piano terra del Dipartimento di Informatica, a fianco del suo ufficio.
\end{tcolorbox}

\subsubsection{Griglia di Valutazione}
La valutazione dell'implementazione del sistema si basa sull'analisi del codice Jolie, sull'uso dei costrutti del linguaggio per la creazione di soluzioni efficienti, sulla tolleranza ai guasti del sistema implementato e sulla gestione degli errori. In particolare la griglia di valutazione usata è la seguente:
\newpage
\begin{table}[ht]
 \centering
 {\small
  \begin{tabular}{|l|l|l|}
   \hline
   \textit{\begin{tabular}[c]{@{}l@{}}Uso dei costrutti\\ di Jolie\end{tabular}} & \begin{tabular}[c]{@{}l@{}}Corretto utilizzo dei costrutti per la gestione\\ del parallelismo, uso di \texttt{execution(concurrent)}\\ a dispetto dell'uso di \texttt{execution(sequential)}\\ in tutti i servizi.\end{tabular} \\ \hline
   \textit{\begin{tabular}[c]{@{}l@{}}Distribuzione del carico\\ di lavoro nel gruppo\end{tabular}} & \begin{tabular}[c]{@{}l@{}}Omogeneità nella ripartizione dei  compiti\\ nel gruppo, ogni membro\\  partecipa egualmente allo sviluppo, assenza\\ di dissimmetria di informazione.\end{tabular} \\ \hline
   \textit{\begin{tabular}[c]{@{}l@{}}Grado di partecipazione\\ alla comunità\end{tabular}} & \begin{tabular}[c]{@{}l@{}}Presenza di domande e risposte sulla\\ newsgroup del corso\end{tabular} \\ \hline
  \end{tabular}
 }
\end{table}

\subsection{La Demo}
La demo, o dimostrazione di funzionamento, esegue e monitora il comportamento di un sistema di gestione decentralizzata di transazioni. Essa simula almeno quattro utenti che entrano a far parte della rete peer-to-peer come nodi, un server timestamp che gestisce la sincronizzazione oraria ed un tool per monitorare lo stato della rete (il Network Visualiser).
