\documentclass{article}

\usepackage{amsmath}
\usepackage{amssymb}
\usepackage[bookmarks=false,colorlinks,urlcolor=blue,%
linkcolor=magenta,citecolor=red,linktocpage=true,breaklinks=true]{hyperref}
\usepackage{graphicx}
\usepackage{tcolorbox}
\usepackage{booktabs}
\renewcommand*\contentsname{Indice}
\usepackage{cleveref}
\usepackage{xcolor}
\usepackage{fancyvrb}
\usepackage{listings}
\usepackage{xparse}
\usepackage{tikz}
\usetikzlibrary{arrows}
\usepackage{pgf-umlsd}

\include{macros}

\title{%
  Jolella -- Condivisione Files in Rete Paritaria\\
  \large Documentazione di Progetto per Laboratorio di Sistemi Operativi}

\author{%
  Tutor: Stefano Pio Zingaro\\
  A.A. 2018-2019}

\date{8 Maggio, 2019}

\begin{document}

\maketitle

\begin{abstract}

 Il progetto propone la creazione un sistema di condivisione di file
 decentralizzato (\textit{file sharing}), in una rete di entità pari tra loro
 (\textit{peer-to-peer}). L'implementazione del progetto segue i principi della
 programmazione orientata ai servizi, rispettandone l'architettura e i paradigmi
 di gestione della concorrenza. I servizi e la comunicazione tra di essi sono
 implementati nel linguaggio di programmazione
 \href{http://jolie-lang.org}{Jolie}.

\end{abstract}

\newpage

\tableofcontents

\newpage

\section{Informazioni Logistiche}
\label{sec:logistic}
% !TEX root =  main.tex

In questa sezione sono riportate le istruzioni sulla formazione dei gruppi.
Viene inoltre proposto un elenco di possibili date per la consegna
dell'implementazione e del report. Tali informazioni sono soggette a cambiamenti
ed a revisioni, ogni modifica viene comunicata attraverso la
\href{https://szingaro.github.io/education/so-lab-cs.html}{pagina web del tutor}
ed il \href{https://groups.google.com/forum/#!forum/infoman-so}{forum ufficiale
del corso}. È possibile chiedere delucidazioni e prenotare un ricevimento via
messaggio di posta elettronica, specificandone la motivazione, a
\href{mailto:stefanopio.zingaro@unibo.it}{questo indirizzo}.

\subsection{Formazione dei Gruppi}

I gruppi sono costituiti da un minimo di quattro (4) ad un massimo di cinque
persone (5), coloro che intendono partecipare all'esame comunicano entro e non
oltre il \textbf{20 Maggio 2018} (pena esclusione dall'esame) la composizione
del gruppo di lavoro, \textbf{via posta elettronica}, a
\href{mailto:stefanopio.zingaro@unibo.it}{questo indirizzo}. Il messaggio ha
come oggetto \textbf{GRUPPO LSO} e contiene:

\begin{enumerate}
 \item Il nome del gruppo;
 \item Una riga per ogni componente: cognome, nome e matricola;
 \item Un indirizzo di posta elettronica di riferimento a cui mandare le notifiche, è incarico del proprietario trasmetterle agli altri membri.
\end{enumerate}

\begin{tcolorbox}[colback=green!20!white,colframe=green!75!black,title=Email di esempio con oggetto \textbf{GRUPPO LSO}]
 \textit{NomeGruppo}
 \begin{itemize}
  \item \textit{Vader Darth, 123456}
  \item \textit{Pallino Pinco, 234567}
  \item \textit{Pannocchia Anna, 345678}
 \end{itemize}
 \textit{Referente: anna.pannocchia@studio.unibo.it}
\end{tcolorbox}

Chi non riuscisse a trovare un gruppo invia allo stesso indirizzo di posta
elettronica un messaggio con oggetto \textbf{CERCO GRUPPO LSO}, specificando:

\begin{enumerate}

 \item Cognome, Nome, Matricola, Email;

 \item Eventuali preferenze legate a luogo e tempi di lavoro (si cercherà di
       costituire gruppi di persone con luoghi e tempi di lavoro compatibili, nel
       limite delle possibilità del tutor).

\end{enumerate}

\begin{tcolorbox}[colback=green!20!white,colframe=green!75!black,title=Email di esempio con oggetto \textbf{CERCO GRUPPO LSO}]
 \textit{Sinalefe Pina, 234567, preferirei nei pressi del dipartimento, tutti i giorni dopo pranzo.}
\end{tcolorbox}

Le persone senza un gruppo vengono assegnate il prima possibile senza
possibilità di ulteriori modifiche.

\subsection{Date di Consegna ed Esame Orale}

Le date a disposizione per la consegna dell'implementazione e del report, sono:

\begin{itemize}
 \item le 23.59.59 di \textbf{Lunedí 1 Luglio} 2019;
 \item le 23.59.59 di \textbf{Lunedí 2 Settembre} 2019.
\end{itemize}

La data presa in considerazione per la consegna della parte di implementazione
sarà quella di creazione del \textbf{Tag} su \href{https://gitlab.com}{GitLab}
le istruzioni sulla consegna si trovano più avanti nella sezione
~\ref{sec:delivery}. Solo in seguito alle consegne, vengono fissate data ed
orario della discussione (notificate tramite la mail di riferimento),
compatibilmente coi tempi di correzione. La discussione dell'implementazione e
la relativa demo di funzionamento viene effettuata in un incontro unico con
tutti i componenti presenti. Al termine della discussione, ad ogni singolo
componente verrà assegnato un voto in base all'effettivo contributo dimostrato
nel lavoro. La valutazione è indipendente dal numero di persone che compongono
il gruppo.


\section{Descrizione delle Componenti del Sistema}
\label{sec:components}
% !TEX root =  main.tex

In questa sezione vengono descritte le componenti del sistema preso in
considerazione per la realizzazione del progetto Jolella. Tali descrizioni hanno
il solo scopo di informare, e non di porre vincoli sulle scelte implementative.
Alcuni di questi vincoli e limitazioni sono invece discussi nella
sezione~\ref{sec:implementation}, insieme alle motivazioni delle stesse.

\subsection{La rete Peer-To-Peer}
\label{subsec:p2p}

Il termine peer-to-peer~\cite{AS04,Balakrishnan:2003:LUD:606272.606299} si
riferisce ad una classe di sistemi ed applicazioni che utilizzano risorse
distribuite per eseguire una funzionalità (critica) in modo
decentralizzato~\cite{peer-to-peer}.

Ogni nodo in una rete P2P viene chiamato generalmente \textbf{peer}, ha la
medesima importanza e svolge le stesse funzioni di tutti gli altri nodi.

Le caratteristiche di un sistema P2P possono essere riassunte nelle seguenti:

\begin{itemize}

 \item i peer sono indipendenti (autonomi);

 \item ogni peer ha funzioni di \textbf{client} e di \textbf{server}
       contemporaneamente, e condivide delle risorse (in questo caso files);

 \item possono essere presenti dei nodi con funzionalità diverse rispetto agli
       altri, ad esempio per il monitoraggio o, in generale, per offrire operazioni
       specifiche;

 \item il sistema è \textbf{altamente distribuito}, il numero di peers può
       essere dell'ordine delle centinaia di migliaia;

 \item il sistema è \textbf{altamente dinamico}, un peer può entrare ed uscire
       dalla rete in ogni momento.

       \textbf{N.B.}

       \begin{itemize}

        \item le operazioni di ingresso/uscita (\textbf{join}/\textbf{leave}) dalla
              rete fanno parte dell'interfaccia di ogni peer.

       \end{itemize}

\end{itemize}

\subsubsection{Architetture di una rete P2P}

Esistono tre tipologie di reti distribuite peer-to-peer:

\begin{enumerate}

 \item Decentralizzate pure -- dove tutti i nodi sono peer, non esiste nessun
       coordinatore centralizzato. Ogni peer può funzionare come client e/o server.

 \item Parzialmente centralizzate -- in queste reti alcuni nodi (supernodi)
       facilitano l’interconnessione tra i peer, questi hanno, di solito, un indice
       locale centralizzato dei peer locali. Il
       pattern di comunicazione è: un peer comunica con il supernodo, che a sua volta
       comunica con gli altri peer.

 \item Decentralizzate ibride -- queste reti presentano un server centralizzato
       che facilita l’interazione tra i peer (servizio di \textbf{localizzazione}, ad
       esempio tramite \textbf{location}).

\end{enumerate}

\paragraph{Una rete P2P per la condivisione di files.} La condivisione di files
costituisce il principale utilizzo delle reti decentralizzate fin dagli anni
`90 (si parla di quasi il $30\%$ della rete Internet). Perchè una rete P2P sia
abilitata alla condivisione di files è necessario che ogni nodo esponga delle
funzionalità di \textbf{pubblicazione}, \textbf{ricerca} e \textbf{recupero}.

Vediamo un ``classico'' esempio di comunicazione tra due entità in una rete
peer-to-peer per la condivisione di un file:

\begin{enumerate}

 \item Alice esegue un \textbf{Client} P2P;

 \item Alice registra il suo contenuto nel sistema P2P;

 \item Alice cerca ``Hey Jude'';

 \item Il \textbf{Client} visualizza altri peer che hanno una copia di ``Hey
       Jude'';

 \item Alice sceglie uno, o più peer, per esempio Bob;

 \item il file viene copiato da Bob ad Alice;

 \item mentre Alice esegue il download, altri peer chiedono ad Alice l'upload
       di file.

\end{enumerate}

L'esempio riportato, senza pretesa di descrivere in modo completo il flusso di
comunicazioni, evidenzia alcune importanti proprietà che un \textbf{Client} di
una rete P2P per la condivisione files deve soddisfare. Esso, infatti, non solo
consente ad Alice di registrare un file per la sua condivisione ma, ``sotto il
tappeto'', si occupa, per esempio, di fare il \textit{join} della rete prima di
tutto il resto. Inoltre, permette ad Alice di copiare files condivisi da altri
utenti della rete e consente di individuare quali peer possiedono un determinato
file, tramite una ricerca basata su \textbf{keyword}.

\paragraph{Sistemi con ricerca flood-based.} Riguardo all'ultimo punto trattato,
cioè quello della ricerca in una rete decentralizzata, tale problema può essere
ricondotto a quello del \textit{routing} o instradamento.


\subsection{Il nodo/peer}


\subsection{Il monitor di rete}


\section{Discussione sulle Strategie di Implementazione}
\label{sec:implementation}
% !TEX root =  main.tex

Nella presente sezione vengono discusse le scelte implementative per la
realizzazione di un sistema come quello descritto nella
sezione~\ref{sec:components}. Le componenti vengono infatti declinate nel caso
specifico di una rete (Jolella) che presenterà differenze (anche se piccole) per
ogni gruppo di studenti coinvolto nella realizzazione di questo progetto.

\subsection{La rete Jolella}
\label{subsec:jolella}

Jolella, ispirata al programma Gnutella~\cite{gnutella}, è una rete
decentralizzata per la condivisione di files tra nodi della rete stessa (anche
detta \textit{P2P})~\footnote{Creato nel 2000, è stato uno dei primi protocolli
 di condivisione di files su una rete decentralizzata peer-to-peer.}.

\paragraph{Architettura di Jolella.} La scelta dell'architettura della rete non è
vincolata, il progetto potrà infatti essere svolto seguendo una delle tre
strategie proposte. Viene invece richiesta coerenza nelle scelte implementative
derivanti dall'architettura selezionata, ed una concisa documentazione sulle
motivazioni che hanno portato a tale scelta.

\subsubsection{Obiettivi e Problemi dei sistemi P2P}

Esistono motivazioni specifiche per cui si è scelto il sistema P2P nella
realizzazione del progetto proposto. Alcune risiedono nelle caratteristiche del
sistema stesso, altre, legate ad esigenze didattiche, mirano a mettere in luce
il livello di apprendimento raggiunto su alcuni concetti teorici specifici (ad
esempio quelli relativi alla \textbf{gestione della concorrenza}).

Tra le caratteristiche del sistema alcune sono interessanti rispetto alle
soluzioni ``centralizzate''. Queste sono l'aumento di scalabilità (la creazione
di un peer e del milionesimo peer è uguale); l'interoperabilità e la possibilità
di aggregazione delle risorse; l'autonomia ed il dinamismo (se un peer si
disconette, la rete non subisce modifiche).

Alcune limitazioni sono imposte sulla strategia di realizzazione del progetto
per innescare l'esigenza di risoluzione del problema proposto e consolidare così
i concetti teorici. Le criticità dei sistemi P2P da considerare, sono (in ordine
sparso): la \textbf{sicurezza e consistenza delle comunicazioni}, la
\textbf{disuguglianza tra nodi} e \textbf{l'elevato dinamismo o instabilità
 della rete}. Una particolare attenzione va riservata a questi aspetti, la cui
risoluzione costituisce il fulcro stesso del progetto.

\paragraph{Altri vincoli e limitazioni.} Di seguito alcuni vincoli che, per
semplicità, vengono imposti sul sistema:

\begin{itemize}

 \item assumiamo che la ricerca per keyword, sia effettuata sui
       nomi dei file, e che questi corrispondano con l'effettivo contenuto
       (non sono \textit{fake files}).

 \item assumiamo che la rete sia una rete \textbf{non strutturata}, cioè
       organizzata come un \textbf{grafo random} e dove non esistono vincoli sul
       posizionamento dei nodi rispetto alla topologia del grafo. In questa
       configurazione l'aggiunta o la rimozione del nodo non comporta la
       riorganizzazione della rete. Per delucidazioni sui grafi consiglio i lucidi presenti in~\cite{grafi}, alla voce ``Grafi''.

 \item in caso di sistemi ad architettura puramente decentralizzata, e quindi
       con ricerca flood-based, si richiede l'utilizzo di un  ID per ogni query, così
       da controllare che questa non sia già stata risolta dal nodo  ricevente.

\end{itemize}

\subsection{Il Jeer}
\label{subsec:jeer}

Un peer della rete Jolella (che, per disambiguità, chiameremo \textbf{Jeer}) si
può occupare sia delle operazioni di gestione della rete, sia delle funzionalità
del singolo peer (descritte entrambe nella sezione~\ref{subsec:p2p}).

In Gnutella, i peer offrono una operazione di \textbf{discovery} che permette ad
un nodo che conosce l'indirizzo di un altro nodo, di accedere alle liste di nodi
connessi. Per la ricerca, il  nodo utilizza un algoritmo flood-based di tipo
\textit{breadth-first} (BFS).

\begin{figure}[H]
 \centering
 \includegraphics[width=0.75\textwidth]{gnutella}
 \caption{Un peer può rifiutare una richiesta di connessione, ad esempio perché ha raggiunto un numero massimo di connessioni ammesse.}
\end{figure}

\subsection{Monitorare Jolella ed i Jeer}

Abbiamo bisogno di un sistema di feedback per il monitoraggio della rete, sia in
fase di \textit{debugging}, che durante la manutenzione del sistema. Inoltre,
durante la prova orale, sarò necessario per il controllo della corretta
implementazione del soluzioni proposte dal gruppo. Il monitor può essere visto
come una console di \textit{logging}, dove è possibile visualizzare lo storico
della rete. Esso può essere stampato a video oppure su un file. Un esempio di
output del monitor della rete potrebbe consistere nel semplice elenco numerato
riportato in basso, \textbf{la numerazione è obbligatoria}.

\begin{verbatim}
1. La rete Jolella è attiva!
2. Il Jeer <id_jeer> ora partecipa alla rete.
3. Il Jeer <id_jeer> ha pubblicato il Jile <id_jile>
\end{verbatim}


\section{Specifiche per la Consegna}
\label{sec:delivery}
% !TEX root =  main.tex

Globalmente, vengono consegnati due prodotti: la documentazione e il codice
sorgente del progetto. La valutazione finale avviene mediante una discussione di
gruppo (orale), nella quale vengono discusse le strategie con la quale i
prodotti consegnati sono stati generati. Le domande possono essere rivolte a
chiunque dei partecipanti al gruppo.

\begin{tcolorbox}[colback=yellow!20!white,colframe=yellow!75!black,title=\textbf{N.B.}]

 \begin{enumerate}

  \item non si accettano richieste di eccezioni sui progetti con motivazioni
        legate a esigenze di laurearsi o di non voler pagare le tasse per un altro
        anno.

  \item chi copia o fa copiare, anche solo in parte, si vede invalidare
        completamente il progetto senza possibilità di appello. Il codice viene
        controllato con un programma per il rilevamento di plagio.

 \end{enumerate}

\end{tcolorbox}

\subsection{La documentazione}

È possibile scrivere la documentazione nel formato preferito, l'importante è che
il PDF generato rispetti la struttura del modello (riportato in
basso~\ref{template}). La documentazione ha lunghezza di quattro o cinque pagine
(quindi da 8 a 10 facciate), è scritto con font di grandezza \textbf{12pt} e
viene consegnato in formato PDF. Di seguito viene riportato un esempio di
documentazione con le principali caratteristiche da inserire.

\begin{tcolorbox}[colback=green!20!white,colframe=green!75!black,title=L'intestazione della Documentazione]
 \label{template}
 \begin{itemize}
  \item Jolella -- Laboratorio Sistemi Operativi A.A. 2018-2019
  \item Nome del Gruppo
  \item Indirizzo mail di riferimento: nome.cognome@studio.unibo.it
  \item Componenti:
        \begin{itemize}
         \item Cognome, Nome, Matr. 0000424242
         \item \dots
        \end{itemize}
 \end{itemize}
\end{tcolorbox}

\begin{tcolorbox}[colback=green!20!white,colframe=green!75!black,title=Il corpo della Documentazione]

 \begin{enumerate}

  \item Descrizione generale del progetto -- descrizione delle features
        implementate e del contenuto della documentazione.

  \item Istruzioni per la demo -- le istruzioni per eseguire una demo.

  \item Discussione sulle strategie di implementazione:

        \begin{enumerate}

         \item Struttura del progetto -- come è stato diviso il progetto,
               perché, i problemi principali riscontrati, le alternative considerate e
               le soluzioni scelte.

         \item Sezione di descrizione della feature x -- abbiamo implementato la
               funzione di `foo` \dots (con esempi di codice).

        \end{enumerate}

 \end{enumerate}

\end{tcolorbox}

\subsubsection{Griglia di Valutazione}

La valutazione della documentazione verte sull'analisi dello scritto e sulla sua
capacità di esprimere con chiarezza i concetti descritti, soprattutto \textbf{grazie
 all'uso di esempi}.

In particolare la griglia di valutazione usata è la seguente:

\begin{table}[ht]
 \centering
 {\small
  \begin{tabular}{|l|l|}
   \hline
   \textit{\begin{tabular}[c]{@{}l@{}}Qualità dell'informazione\end{tabular}} & \begin{tabular}[c]{@{}l@{}}Riconoscimento dei problemi (di concorrenza)\\e loro descrizione.\end{tabular} \\ \hline
   \textit{\begin{tabular}[c]{@{}l@{}}Uso degli esempi\end{tabular}} & \begin{tabular}[c]{@{}l@{}}Presenza di almeno un esempio in tutte le scelte\\implementative.\end{tabular} \\ \hline
   \textit{\begin{tabular}[c]{@{}l@{}}Analisi delle scelte \\ implementative\end{tabular}} & \begin{tabular}[c]{@{}l@{}}Descrizione  della propria scelta implementativa e\\presenza di proposte di alternative valide.\end{tabular} \\ \hline
  \end{tabular}
 }
\end{table}

\subsection{Il codice sorgente}

Il progetto viene sviluppato utilizzando il linguaggio Jolie. Non ci sono
requisiti riguardo ai protocolli (\textit{protocol}) e i media
(\textit{location}) utilizzati per realizzare la comunicazione tra i componenti
del sistema.

La gestione del progetto avviene col supporto del sistema \textit{git}, a
\href{https://education.github.com/git-cheat-sheet-education.pdf}{questa pagina} è
possibile trovare una lista di comandi utili da tenere sempre a mente.

Il codice del progetto è contenuto in un repository sul server in cloud del
servizio online \href{http://gitlab.com}{GitLab} e viene gestito seguendo la
procedura qui descritta.

\begin{tcolorbox}[colback=blue!20!white,colframe=blue!75!black,title=GitLab]
 \begin{itemize}
  \item Ogni membro del gruppo crea un account su GitLab;
  \item il referente del gruppo  crea un nuovo progetto cliccando sul \textbf{+} in alto a destra nella schermata principale di GitLab, inserendo il nome ``LabSO\_NomeGruppo'' e cliccando su \textbf{New Project};
  \item una volta che il progetto è stato creato, il referente aggiunge ogni membro del gruppo come \textbf{role permission} $>$\textbf{Developer} al progetto andando su \textbf{Settings} $>$ \textbf{Members} nel menù a sinistra, cercandoli in base allo username registrato su GitLab;
  \item il referente aggiunge l'utente ``stefanopiozingaro'' come \textbf{role permission} $>$ \textbf{Reporter}.
 \end{itemize}
\end{tcolorbox}

\subsubsection{La consegna dell'implementazione tramite il Tag di GitLab}

Al momento della consegna, il repository dovrà contenere i sorgenti del progetto
e la relazione, nominata \textbf{REPORT\_LSO.pdf}. Per effettuare la consegna:

\begin{enumerate}

 \item nella pagina di GitLab del repository, cliccare sulle voci del menù
       \textbf{Repository} $>$ \textbf{Tags} $>$ \textbf{New Tag};

 \item digitare come \textbf{Tag Name} il nome \textbf{Consegna};

 \item cliccare su \textbf{Create Tag} per eseguire la creazione del
       \textbf{Tag} di consegna.

\end{enumerate}

Una volta creato il Tag, inviare una email di notifica di consegna con soggetto
\textbf{CONSEGNA LSO - NOME GRUPPO} a
\href{mailto:stefanopio.zingaro@unibo.it}{questo indirizzo di posta
 elettronica}.

\begin{tcolorbox}[colback=yellow!20!white,colframe=yellow!75!black,title=\textbf{N.B.}]

 Inoltre, La documentazione va consegnata in forma cartacea nella casella del
 Professor Sangiorgi al piano terra del Dipartimento di Informatica.

\end{tcolorbox}

\subsubsection{Griglia di Valutazione}

La valutazione dell'implementazione del sistema si basa sull'analisi del codice
Jolie, sull'uso dei costrutti del linguaggio per la creazione di soluzioni
efficienti, sulla tolleranza ai guasti del sistema implementato e sulla gestione
degli errori. In particolare la griglia di valutazione usata è la seguente:

\begin{table}[ht]
 \centering
 {\small
  \begin{tabular}{|l|l|l|}
   \hline
   \textit{\begin{tabular}[c]{@{}l@{}}Uso dei costrutti\\ di Jolie\end{tabular}} & \begin{tabular}[c]{@{}l@{}}Corretto utilizzo dei costrutti per la gestione\\ della concorrenza, uso corretto di \texttt{execution(\ldots)}.\end{tabular} \\ \hline
   \textit{\begin{tabular}[c]{@{}l@{}}Distribuzione del carico\\ di lavoro nel gruppo\end{tabular}} & \begin{tabular}[c]{@{}l@{}}Omogeneità nella ripartizione dei compiti\\ nel gruppo, ogni membro\\  partecipa egualmente allo sviluppo, indica \\il singolo contributo assenza\\ di dissimmetria di informazione.\end{tabular} \\ \hline
   \textit{\begin{tabular}[c]{@{}l@{}}Grado di partecipazione\\ alla comunità\end{tabular}} & \begin{tabular}[c]{@{}l@{}}Presenza di domande e risposte sul\\ forum del corso\end{tabular} \\ \hline
  \end{tabular}
 }
\end{table}

\subsection{Il test del progetto}

Insieme alla documentazione ed al codice sorgente, dovrà essere preparato uno
script che permette di automatizzare i test. Almeno nella fase iniziale della
prova orale, può essere utile preparare una serie di \textit{screenshot}
correlati allo script, che permetteranno di velocizzare le operazioni di
controllo del codice. Tale suite si test può essere intergrata nel codice
sorgente


\bibliographystyle{ieeetr}
\bibliography{biblio}

\end{document}
