\documentclass{article}

\usepackage{amsmath}
\usepackage{amssymb}
\usepackage[bookmarks=false,colorlinks,urlcolor=blue,%
linkcolor=magenta,citecolor=red,linktocpage=true,breaklinks=true]{hyperref}
\usepackage{graphicx}
\usepackage{tcolorbox}
\usepackage{booktabs}
\renewcommand*\contentsname{Indice}
\usepackage{cleveref}
\usepackage{xcolor}
\usepackage{fancyvrb}
\usepackage{listings}
\usepackage{xparse}
\usepackage{tikz}
\usetikzlibrary{arrows}
\usepackage{pgf-umlsd}

\usepackage{bold-extra}

\newcommand{\Jolie}{Jolie}
\newcommand{\Definition}{\noindent\textbf{\emph{Definition}}}
\newcommand{\Implementation}{\noindent\textbf{\emph{Implementation}}}
\newcommand{\Section}{\S}

\newcommand{\citeNeed}{{\color{red}[CitNeed]}}

\definecolor{color:keyword}{rgb}{0.53,0.05,0.05}
\definecolor{color:comment}{rgb}{0.25,0.37,0.75}
\definecolor{color:string}{rgb}{0.87,0.0,0.0}

\lstdefinelanguage{Jolie}{
    morekeywords={csets,type,raw,any,undefined,void,default,if,for,while,spawn,foreach,else,define,main,include,constants,inputPort,outputPort,interface,execution,cset,nullProcess,RequestResponse,OneWay,throw,throws,install,scope,embedded,init,synchronized,global,is_defined,is_int,is_bool,is_long,is_string,bool,long,int,string,double,undef,with,Location,Protocol,Interfaces,Aggregates,Redirects,linkIn,linkOut},
    sensitive=true,
    morecomment=[l]{//},
    morecomment=[s]{/*}{*/},
    morestring=[b]",
    otherkeywords={;,|,@}
}

\lstset{
    language=Jolie,
    mathescape=true,
    resetmargins=true,
    numberstyle=\footnotesize,
    numbers=none,
    numbersep=5pt,
    numberblanklines=true,
    basicstyle=\ttfamily\small,
    tabsize=2,
    %frame=lines,
    commentstyle=\rmfamily\color{color:comment},
    stringstyle=\color{color:string},
    captionpos=b,
    keywordstyle=\bfseries\color{color:keyword},
    showstringspaces=false,
    belowcaptionskip=10mm,
    breaklines=false,
    columns=fullflexible,
    linewidth= 0.8\linewidth
}

\newcommand{\code}[1]{\lstinline{#1}{}}

\newcommand{\setUmlSeqChartStyle}{
    \tikzset{inststyle/.style={
    rectangle, draw, 
    anchor=west, 
    minimum height=0.8cm, 
    minimum width=1.6cm, 
    fill=white
    %drop shadow={opacity=0,fill=black}]
    }
  }
}

\renewcommand{\mess}[4][0]{
  \stepcounter{seqlevel}
  \path
  (#2)+(0,-\theseqlevel*\unitfactor-0.7*\unitfactor) node (mess from) {};
  \addtocounter{seqlevel}{#1}
  \path
  (#4)+(0,-\theseqlevel*\unitfactor-0.7*\unitfactor) node (mess to) {};
  \draw[->,>=angle 60] (mess from) -- (mess to)% 
    node[midway, above, align=center, text width=3cm]
    {\footnotesize #3};

  \node (#3 from) at (mess from) {};
  \node (#3 to) at (mess to) {};
}

\crefname{figure}{Fig.}{Figs.}
\crefname{lstlisting}{Listing}{Listings}


\title{%
  Jolella -- Condivisione Files in Rete Paritaria\\
  \large Documentazione di Progetto per Laboratorio di Sistemi Operativi}

\author{%
  Tutor: Stefano Pio Zingaro\\
  A.A. 2018-2019}

\begin{document}

\maketitle
\date{}

\begin{abstract}
\noindent Il progetto di quest'anno propone la creazione un sistema di scambio elettronico decentralizzato, dove gli utenti effettuano transazioni certificate all'interno della rete stessa, senza il bisogno di un organo centrale garante\footnote{Il ruolo che ricoprono le banche nei moderni sistemi finanziari.}. L'implementazione del progetto segue i principi della programmazione orientata ai servizi, rispettandone l'architettura e i paradigmi di gestione della concorrenza. La comunicazione tra i servizi, infine, è implementata nel linguaggio visto a lezione di laboratorio: \href{http://jolie-lang.org}{Jolie}~\cite{milner1992calculus}.
\end{abstract}

\newpage

\tableofcontents

\newpage

\section{Informazioni Logistiche}
\label{sec:logistic}
% !TEX root =  main.tex

In questa sezione sono riportate le istruzioni sulla formazione dei gruppi.
Viene inoltre proposto un elenco di possibili date per la consegna
dell'implementazione e del report. Tali informazioni sono soggette a cambiamenti
ed a revisioni, ogni modifica viene comunicata attraverso la
\href{https://szingaro.github.io/}{pagina web del tutor} ed il
\href{https://groups.google.com/forum/#!forum/infoman-so}{forum ufficiale del
 corso}. È possibile chiedere delucidazioni e prenotare un ricevimento via
messaggio di posta elettronica, specificandone la motivazione, a
\href{mailto:stefanopio.zingaro@unibo.it}{questo indirizzo}.

\subsection{Formazione dei Gruppi}

I gruppi sono costituiti da un minimo di quattro (4) ad un massimo di cinque
persone (5), coloro che intendono partecipare all'esame comunicano entro e non
oltre il \textbf{20 Maggio 2018} (pena esclusione dall'esame) la composizione
del gruppo di lavoro, \textbf{via posta elettronica}, a
\href{mailto:stefanopio.zingaro@unibo.it}{questo indirizzo}. Il messaggio ha
come oggetto \textbf{GRUPPO LSO} e contiene:

\begin{enumerate}
 \item Il nome del gruppo;
 \item Una riga per ogni componente: cognome, nome e matricola;
 \item Un indirizzo di posta elettronica di riferimento a cui mandare le notifiche, è incarico del proprietario trasmetterle agli altri membri.
\end{enumerate}

\begin{tcolorbox}[colback=green!20!white,colframe=green!75!black,title=Email di esempio con oggetto \textbf{GRUPPO LSO}]
 \textit{NomeGruppo}
 \begin{itemize}
  \item \textit{Vader Darth, 123456}
  \item \textit{Pallino Pinco, 234567}
  \item \textit{Banana Joe, 345678}
 \end{itemize}
 \textit{Referente: joe.banana@studio.unibo.it}
\end{tcolorbox}

Chi non riuscisse a trovare un gruppo invia allo stesso indirizzo di posta
elettronica un messaggio con oggetto \textbf{CERCO GRUPPO LSO}, specificando:

\begin{enumerate}
 \item Cognome, Nome, Matricola, Email;
 \item Eventuali preferenze legate a luogo e tempi di lavoro (si cercherà di costituire gruppi di persone con luoghi e tempi di lavoro compatibili, nel limite delle possibilità del tutor).
\end{enumerate}

\begin{tcolorbox}[colback=green!20!white,colframe=green!75!black,title=Email di esempio con oggetto \textbf{CERCO GRUPPO LSO}]
 \textit{Pallino Pinco, 234567, preferirei nei pressi del dipartimento, tutti i giorni dopo pranzo.}
\end{tcolorbox}

Le persone senza un gruppo vengono assegnate il prima possibile senza
possibilità di ulteriori modifiche.

\subsection{Date di Consegna ed Esame Orale}

Le date a disposizione per la consegna dell'implementazione e del report, sono:

\begin{itemize}
 \item le 23.59.59 di \textbf{Lunedí 1 Luglio} 2019;
 \item le 23.59.59 di \textbf{Lunedí 2 Settembre} 2019.
\end{itemize}

La data presa in considerazione per la consegna della parte di implementazione
sarà quella di creazione del \textbf{Tag} su \href{https://gitlab.com}{GitLab}
le istruzioni sulla consegna si trovano più avanti nella sezione
~\ref{sec:delivery}. Solo in seguito alle consegne, vengono fissate data ed
orario della discussione (notificate tramite la mail di riferimento),
compatibilmente coi tempi di correzione. La discussione dell'implementazione e
la relativa demo di funzionamento viene effettuata in un incontro unico con
tutti i componenti presenti. Al termine della discussione, ad ogni singolo
componente verrà assegnato un voto in base all'effettivo contributo dimostrato
nel lavoro. La valutazione è indipendente dal numero di persone che compongono
il gruppo.


\newpage

\section{Componenti del Progetto e loro Implementazione}
\label{sec:components}
% !TEX root =  main.tex

\subsection{La rete Peer-To-Peer -- Jolella}

\subsection{Il nodo/peer -- il Jeer}

\subsection{Il monitoraggio della rete -- il Network Monitor}


\newpage

\section{Specifiche di Consegna}
\label{sec:delivery}
Globalmente vengono consegnati tre prodotti: la documentazione del progetto, la sua implementazione in codice Jolie ed una demo di funzionamento. La valutazione finale avviene mediante una discussione di gruppo.
\begin{tcolorbox}[colback=yellow!20!white,colframe=yellow!75!black,title=\textbf{N.B.}]
    \begin{enumerate}
        \item non si accettano richieste di eccezioni sui progetti con motivazioni legate a esigenze di laurearsi o di non voler pagare le tasse per un altro anno.
        \item chi copia o fa copiare, anche solo in parte, si vede invalidare completamente il progetto senza possibilità di appello $\rightarrow$ deve quindi rifare un nuovo progetto l'anno successivo.
    \end{enumerate}
\end{tcolorbox}

\subsection{La documentazione}
È possibile scrivere la documentazione nel formato preferito, l'importante è che il PDF generato rispetti la struttura del modello (riportato qui in basso). La documentazione ha lunghezza di quattro o cinque pagine (quindi da 8 a 10 facciate), è scritto con font di grandezza \textbf{12pt} e viene consegnato in formato PDF. 
Di seguito viene riportato un esempio di documentazione con le principali caratteristiche da inserire.
\begin{tcolorbox}[colback=green!20!white,colframe=green!75!black,title=L'intestazione della Documentazione]
\begin{itemize}
    \item Jollar -- Laboratorio Sistemi Operativi A.A. 2017-2018
    \item Nome del Gruppo
    \item Indirizzo mail di riferimento: myaccount@email.com
    \item Componenti:
    \begin{itemize}
        \item Cognome, Nome, Matr. 0000424242
        \item \dots
    \end{itemize}
\end{itemize}
\end{tcolorbox}
\begin{tcolorbox}[colback=green!20!white,colframe=green!75!black,title=Il corpo della Documentazione]
\begin{enumerate}
    \item Descrizione generale del progetto -- descrizione delle features implementate e del contenuto della documentazione.
    \item Istruzioni per la demo -- le istruzioni per eseguire una demo.
    \item Discussione sulle strategie di implementazione 
    \begin{enumerate}
        \item Struttura del progetto -- come è stato diviso il progetto, perché, i problemi principali riscontrati, le alternative considerate e le soluzioni scelte.
        \item Sezione di descrizione della feature x -- abbiamo implementato la funzione di `foo` \dots (con esempi di codice).
    \end{enumerate}
\end{enumerate}
\end{tcolorbox}

\subsubsection{Griglia di Valutazione}
La valutazione della documentazione verte sull'analisi dello scritto e sulla sua capacità di esprimere con chiarezza i concetti descritti, soprattutto grazie all'uso di esempi. In particolare la griglia di valutazione usata è la seguente:
\newpage
\begin{table}[ht]
    \centering
    {\small
    \begin{tabular}{|l|l|}
    \hline  
    \textit{\begin{tabular}[c]{@{}l@{}}Qualità dell'informazione\end{tabular}}               & \begin{tabular}[c]{@{}l@{}}Riconoscimento dei problemi (di concorrenza)\\e loro descrizione.\end{tabular}                            \\ \hline
    \textit{\begin{tabular}[c]{@{}l@{}}Uso degli esempi\end{tabular}}                       & \begin{tabular}[c]{@{}l@{}}Presenza di almeno un esempio in tutte le scelte\\implementative.\end{tabular}                                          \\ \hline
    \textit{\begin{tabular}[c]{@{}l@{}}Analisi delle scelte \\ implementative\end{tabular}} & \begin{tabular}[c]{@{}l@{}}Descrizione  della propria scelta implementativa e\\presenza di proposte di alternative valide.\end{tabular} \\ \hline
    \end{tabular}
    }
\end{table}

\subsection{L'Implementazione}
Il progetto viene sviluppato utilizzando il linguaggio Jolie. Non ci sono requisiti riguardo ai protocolli (\textit{protocol}) e i media (\textit{location}) utilizzati per realizzare la comunicazione tra i componenti del sistema. La gestione del progetto avviene col supporto del sistema \textit{git}. Il codice del progetto è contenuto in un repository su \href{http://gitlab.com}{GitLab} e viene gestito seguendo la procedura descritta sotto.
%
\begin{tcolorbox}[colback=blue!20!white,colframe=blue!75!black,title=GitLab]
\begin{itemize}
    \item Ogni membro del gruppo crea un account su GitLab;
    \item il referente del gruppo  crea un nuovo progetto cliccando sul \textbf{+} in alto a destra nella schermata principale di GitLab, inserendo il nome ``LabSO\_NomeGruppo'' e cliccando su \textbf{New Project};
    \item una volta che il progetto è stato creato, il referente aggiunge ogni membro del gruppo come \textbf{role permission} $>$\textbf{Developer} al progetto andando su \textbf{Settings} $>$ \textbf{Members} nel menù a sinistra, cercandoli in base allo username registrato su GitLab;
    \item il referente aggiunge l'utente ``stefanopiozingaro'' come \textbf{role permission} $>$ \textbf{Reporter}.
\end{itemize}
\end{tcolorbox}
%
Al momento della consegna, il repository dovrà contenere i sorgenti del progetto e la relazione, nominata \textbf{REPORT\_LSO.pdf}. Per effettuare la consegna:
\begin{enumerate}
    \item nella pagina di GitLab del repository, cliccare sulle voci del menù \textbf{Repository} $>$ \textbf{Tags} $>$ \textbf{New Tag}; 
    \item digitare come \textbf{Tag Name} il nome \textbf{Consegna};
    \item cliccare su \textbf{Create Tag} per eseguire la creazione del \textbf{Tag} di consegna.
\end{enumerate}
%
Una volta creato il Tag, inviare una email di notifica di consegna con soggetto \textbf{CONSEGNA LSO - NOME GRUPPO} a \href{stefanopio.zingaro@unibo.it}{questo indirizzo di posta elettronica}. 
\begin{tcolorbox}[colback=yellow!20!white,colframe=yellow!75!black,title=\textbf{N.B.}]
    La documentazione va consegnata in forma cartacea nella casella del prof. Sangiorgi al piano terra del Dipartimento di Informatica, a fianco del suo ufficio.
\end{tcolorbox}

\subsubsection{Griglia di Valutazione} 
La valutazione dell'implementazione del sistema si basa sull'analisi del codice Jolie, sull'uso dei costrutti del linguaggio per la creazione di soluzioni efficienti, sulla tolleranza ai guasti del sistema implementato e sulla gestione degli errori. In particolare la griglia di valutazione usata è la seguente:
\newpage
\begin{table}[ht]
    \centering
    {\small
    \begin{tabular}{|l|l|l|}
    \hline
    \textit{\begin{tabular}[c]{@{}l@{}}Uso dei costrutti\\ di Jolie\end{tabular}} & \begin{tabular}[c]{@{}l@{}}Corretto utilizzo dei costrutti per la gestione\\ del parallelismo, uso di \texttt{execution(concurrent)}\\ a dispetto dell'uso di \texttt{execution(sequential)}\\ in tutti i servizi.\end{tabular} \\ \hline
    \textit{\begin{tabular}[c]{@{}l@{}}Distribuzione del carico\\ di lavoro nel gruppo\end{tabular}} & \begin{tabular}[c]{@{}l@{}}Omogeneità nella ripartizione dei  compiti\\ nel gruppo, ogni membro\\  partecipa egualmente allo sviluppo, assenza\\ di dissimmetria di informazione.\end{tabular} \\ \hline
    \textit{\begin{tabular}[c]{@{}l@{}}Grado di partecipazione\\ alla comunità\end{tabular}} & \begin{tabular}[c]{@{}l@{}}Presenza di domande e risposte sulla\\ newsgroup del corso\end{tabular} \\ \hline
    \end{tabular}
    }
\end{table}

\subsection{La Demo}
La demo, o dimostrazione di funzionamento, esegue e monitora il comportamento di un sistema di gestione decentralizzata di transazioni. Essa simula almeno quattro utenti che entrano a far parte della rete peer-to-peer come nodi, un server timestamp che gestisce la sincronizzazione oraria ed un tool per monitorare lo stato della rete (il Network Visualiser).
La sequenza di esecuzione è la seguente:
\begin{enumerate}
    \item avvio del Server Timestamp;
    \item avvio del Network Visualizer;
    \item avvio del primo nodo, generatore del primo blocco (con \textit{reward} di sei Jollar);
    \item avvio del secondo, terzo e quarto nodo (con conseguente \textit{download} della blockchain);
    \item invio di un Jollar dal primo al secondo nodo (con conseguente scrittura su un nuovo blocco e reward relativo);
    \item invio di due Jollar dal primo al terzo nodo (con conseguente scrittura su un nuovo blocco e reward relativo);
    \item invio di tre Jollar dal primo al quarto nodo (con conseguente scrittura su un nuovo blocco e reward relativo);
    \item richiesta al Network Visualiser di stampare la situazione relativa al totale dei Jollar presenti nella rete e dei relativi possessori, con l'elenco delle transazioni avvenute.
\end{enumerate}

\newpage

\bibliographystyle{apalike}
\bibliography{biblio}

\end{document}
